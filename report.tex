\documentclass{article}
\title{COMP2123 self-learning report \\
Unit testing}

\usepackage{sourcecodepro}
\usepackage{fontenc}
\usepackage{xcolor}
\usepackage{listings}
\usepackage{cite}
\usepackage[margin=1in]{geometry}
\usepackage{scrextend}
\usepackage{hyperref}
\hypersetup{
	colorlinks=true,
	linkcolor=blue,
	linktoc=page
}
\definecolor{mGreen}{rgb}{0,0.6,0}
\definecolor{mGray}{rgb}{0.5,0.5,0.5}
\definecolor{mPurple}{rgb}{0.58,0,0.82}
\definecolor{backgroundColour}{rgb}{0.95,0.95,0.92}
\definecolor{remColor}{rgb}{0.8,0.4,0.4}
\lstdefinestyle{Cpp}{
	backgroundcolor=\color{backgroundColour},
	commentstyle=\color{mGreen},
	keywordstyle=\color{magenta},
	numberstyle=\tiny\color{mGray},
	stringstyle=\color{mPurple},
	basicstyle=\footnotesize\ttfamily,
	breakatwhitespace=false,
	breaklines=true,
	captionpos=b,
	keepspaces=true,
	numbers=left,
	numbersep=5pt,
	showspaces=false,
	showstringspaces=false,
	showtabs=false,
	tabsize=4,
	language=C++
}

\def \bs {\textbackslash}

\def \cpp #1 {
	\lstinputlisting[style=Cpp]{cpp/#1}
}
\def \cpprng #1#2#3 {
	\lstinputlisting[style=Cpp, firstline=#2, lastline=#3]{cpp/#1}
}

\def \rem #1 {\iftrue{
	\textcolor{remColor}{
		\begin{addmargin}{2em}
			\small{Remark: #1}
		\end{addmargin}
	}
} \fi}



\begin{document}
\maketitle
\begin{abstract}
	This report goes through the motivation, frameworks used and difficulties of unit testing.
\end{abstract}
\newpage
\tableofcontents
\newpage

\section{Motivation}
As the scale of a software project grows, debugging becomes more complicated.
It may take a long time to discover edge case bugs in an old component, which is very difficult to debug after a long time.
Unit testing allows identification of bugs as soon as possible with little impact.

\section{Unit testing methods}
\subsection{Testing for expected result}
The intuitive way is to write a test that tests each function.

\begin{lstlisting}[style=CppStyle]
class SimpleSpec {
public:
	void testFooBar() {
		ASSERT_EQUAL(fooBar(), "qux")
	}
}
\end{lstlisting}

The \texttt{ASSERT\_EQUAL} macro function would compare the result of \texttt{fooBar()} with \texttt{"qux"} and trigger an error if they are not equal.

\subsection{Generating test parameters}

\subsection{Testing for edge cases}

\subsection{Test case selection}

\section{Unit testing tools}
\subsection{Test coverage}
Test coverage is a criterion to assess the representativeness of the unit tests of a project by counting the number of lines executed in the test.

\subsection{Behaviour-driven development}
\rem{Refer to cucumber}

\section{Modular coupling}

\subsection{Dependency mocking}
\rem{https://enterprisecraftsmanship.com/2016/06/09/styles-of-unit-testing/ provides some insight on why mocking is bad}

\end{document}
